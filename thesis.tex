\documentclass[openany]{book}

\usepackage{CJK}
% \usepackage{url}

\usepackage{indentfirst}
\usepackage{comment}
\usepackage[pdftex]{graphicx}
\usepackage{caption}
\usepackage{subcaption}
\usepackage{amsmath}
\usepackage{verbatim}
\usepackage{setspace}
\usepackage{caption}
\usepackage{slashbox}
\usepackage{fancyhdr}

\usepackage{layout}

% 中文章节号相关包
\usepackage{titlesec,titletoc}
\usepackage{CJKnumb}

%%%%%%%%%%%%%%%%%%%%%%%%%%%%%%%%%%%%%%%%%%%%%%%%%%
% 设置页面格式
%%%%%%%%%%%%%%%%%%%%%%%%%%%%%%%%%%%%%%%%%%%%%%%%%%
% 页边距 上3.5cm 下4.0cm 左、右2.8cm
\usepackage[lmargin=2.8cm,rmargin=2.8cm,tmargin=3.5cm,bmargin=4cm]{geometry}

% 设置文档内链接引用
\usepackage{hyperref}
\hypersetup{
    backref,
    CJKbookmarks, % This is a must
    colorlinks,
    linkcolor = blue,
    citecolor = blue,
    urlcolor  = blue,
}

\pagestyle{fancy}
\fancyhf{}

%%%%%%%%%%%%%%%%%%%%%%%%%%%%%%%%%%%%%%%%%%%%%%%%%%%%
% 字体、字号相关设置
%%%%%%%%%%%%%%%%%%%%%%%%%%%%%%%%%%%%%%%%%%%%%%%%%%%%
% 定义字体
\newcommand{\song}{\CJKfamily{song}}
\newcommand{\kai}{\CJKfamily{gkai}}
\newcommand{\hei}{\CJKfamily{hei}}

% 定义字号
\newcommand{\setfontsize}[2]{\setstretch{#2}\fontsize{#1}{#1}\selectfont}
\newcommand{\fontone}{26pt}        % 一号
\newcommand{\fonttwo}{22pt}        % 二号
\newcommand{\fontsmalltwo}{18pt}   % 小二
\newcommand{\fontthree}{16pt}      % 三号
\newcommand{\fontsmallthree}{15pt} % 小三
\newcommand{\fontfour}{14pt}       % 四号
\newcommand{\fonthalfsmallfour}{13pt} % 半小四
\newcommand{\fontsmallfour}{12pt}  % 小四
\newcommand{\fontbigfive}{11pt}    % 大五
\newcommand{\fontfive}{10.5pt}     % 五号

\captionsetup{tablewithin=none}
\captionsetup{figurewithin=none}

\begin{document}
\begin{CJK*}{UTF8}{song}


%%%%%%%%%%%%%%%%%%%%%%%%%%%%%%%%%%%%%%%%%%%%%%%%%%%%%
% 章节号、章节标题相关设置
%%%%%%%%%%%%%%%%%%%%%%%%%%%%%%%%%%%%%%%%%%%%%%%%%%%%%
% 设置中文章节号
\renewcommand{\contentsname}{目~~录}
\renewcommand{\bibname}{参~考~文~献}
\renewcommand{\listfigurename}{图~例~列~表}
\renewcommand{\listtablename}{表~格~列~表}
\newcommand{\chaptertitlefont}{\hei\bf\setfontsize{\fontthree}{1}}
\newcommand{\sectiontitlefont}{\hei\bf\setfontsize{\fontfour}{1}}
\newcommand{\subsecttitlefont}{\hei\bf\setfontsize{\fontsmallfour}{1}}
\titleformat{\chapter}{\centering}{\chaptertitlefont 第\,\CJKnumber{\thechapter}\,章}{1em}{\chaptertitlefont}{}
\titleformat{\section}{}{\sectiontitlefont \thesection}{1em}{\sectiontitlefont}{}
\titleformat{\subsection}{}{\subsecttitlefont \thesubsection}{1em}{\subsecttitlefont}{}
\titleformat{\subsubsection}{}{}{1em}{\subsecttitlefont}{}

%%%%%%%%%%%%%%%%%%%%%%%%%%%%%%%%%%%%%%%%%%%%%%%%%%%
% 图、表相关设置
%%%%%%%%%%%%%%%%%%%%%%%%%%%%%%%%%%%%%%%%%%%%%%%%%%%
% 图中标注全部用英文
% 图、表引用中有中文,在正文部分重定义

% command: myfig
% parameter:
%    1 - floating pattern (optional)
%    2 - figure path
%    3 - scale factor
%    4 - label
%    5 - short caption name
%    6 - Chinese caption
%    7 - English caption
\newcommand{\myfig}[7][h!]{ %
\begin{figure}[#1]
    \centering
    \includegraphics[scale=#3]{#2}
    \caption[#5]{\kai \label{figure:#4}#6}{Fig.~\thefigure:~#7}
\end{figure}
}

% environment: mytab
% parameter:
%    1 - floating pattern (optional)
%    2 - label
%    3 - short caption name
%    4 - Chinese caption
%    5 - English caption
%    6 - column definition
\newenvironment{mytab}[6][h!]{ %
\begin{table}[#1]
    \centering
    \caption[#3]{\kai \label{table:#2}#4}{\vspace{-8pt} Tab.~\thetable:~#5\\}
    \vspace{4pt}
    \begin{tabular}{#6}
}{ %
    \end{tabular}
\end{table}}

\newcommand{\figref}[1]{图\ref{figure:#1}}
\newcommand{\tabref}[1]{表\ref{table:#1}}

% 图、表标题中文化
\renewcommand{\tablename}{\kai 表}
\renewcommand{\figurename}{\kai 图}

% 章节引用相关
\newcommand{\seclabel}[1]{\label{sec:#1}}
\newcommand{\secref}[1]{\ref{sec:#1}~节}

%%%%%%%%%%%%%%%%%%%%%%%%%%%%%%%%%%%%%%%%%%%%%%%%%%%%%%
% 正文设置
%%%%%%%%%%%%%%%%%%%%%%%%%%%%%%%%%%%%%%%%%%%%%%%%%%%%%%
% 四号、1.25倍间距
\setfontsize{\fontfour}{1.25}
% 开头空两格
\setlength{\parindent}{2em}

%%%%%%%%%%%%%%%%%%%%%%%%%%%%%%%%%%%%%%%%%%%%%%%%%%%%%%
% header and footer
%%%%%%%%%%%%%%%%%%%%%%%%%%%%%%%%%%%%%%%%%%%%%%%%%%%%%%
\renewcommand{\headrulewidth}{0.5pt}
\renewcommand{\footrule}{ %
    {\rule{\headwidth}{1.2pt}} %
    {\rule[0.8\baselineskip]{\headwidth}{0.5pt}}\vskip-0.8\baselineskip
}
\renewcommand{\chaptermark}[1]{\markboth{~第\CJKnumber{\thechapter}章~~~#1~}{}}
    \fancypagestyle{plain}{
    \lhead{上海交通大学硕士学位论文}
    \rhead{\leftmark}
    \cfoot{第~~\thepage~~页}
}
\lhead{上海交通大学硕士专业学位论文}
\rhead{\leftmark}
\cfoot{第~~\thepage~~页}


%%%%%%%%%%%%%%%%%%%%%%%%%%%%%%%%%%%%%%%%%%%%%%%%%%%%%%%%%%%
% main body
%%%%%%%%%%%%%%%%%%%%%%%%%%%%%%%%%%%%%%%%%%%%%%%%%%%%%%%%%%%
\title{多云存储平台与多设备间\\数据备份与同步的研究}
\author{邱爽}
\date{}
\maketitle

% \layout

% 导言部分
\frontmatter
\pagenumbering{Roman}
\input{abstract}

\tableofcontents
\listoffigures
\listoftables

\mainmatter
\input{introduction}
\input{backup_rosycloud}
\input{security_clouddepot}
\input{dedup}
\input{implementation}
\input{evaluation}
\input{conclusion}

\newpage

\bibliographystyle{abbrv}
\bibliography{backup}

\clearpage
\end{CJK*}
\end{document}
